\documentclass[11pt]{article}

\usepackage{fancyhdr}
\usepackage[margin=.9in]{geometry}
\usepackage{amsmath}
\usepackage{enumitem}

\linespread{1.3}
\setlength{\parindent}{0pt}

% ===========================================================================
% Header / Footer
% ===========================================================================
\pagestyle{fancy}
\lhead{\scriptsize  CSC 212: Data Structures and Abstractions - Spring 2019}\chead{}\rhead{\scriptsize Weekly Problem Set \#1}
\lfoot{}\cfoot{\scriptsize \thepage~of~\pageref{r:lastpage}}\rfoot{}
\renewcommand{\headrulewidth}{0.4pt}
\renewcommand{\footrulewidth}{0.4pt}

% ===========================================================================
% ===========================================================================
\begin{document}
\thispagestyle{empty}

% ===========================================================================
\begin{center}
    {\Large\bf CSC 212: Data Structures and Abstractions}\\
    \medskip
    {\Large\bf University of Rhode Island}\\
    \bigskip
    {\Large\bf Spring 2019}\\
    \bigskip
    {\Large\bf Weekly Problem Set \#1}
\end{center}

\vspace{25pt}

This assignment is due Thursday 2/7 before lecture.  Please turn in neat, and organized, answers hand-written on standard-sized paper {\bf without any fringe}.

\begin{enumerate}[leftmargin=*]

\item Provide a sequence of Bash commands that will:
    \begin{itemize}
        \item go to your default home directory;
        \item create a directory test;
        \item rename test to myproject;
        \item enter the directory myproject;
        \item create a new empty file main.c;
        \item list all files in myproject, including hidden files;
        \item return to the parent directory.
    \end{itemize}

\item Provide a sequence of Bash commands that will:
    \begin{itemize}
        \item create files a.txt, b.txt, and c.txt;
        \item write the line a: 1 2 3 4 5 to a.txt;
        \item write the line b: 6 7 8 9 10 to b.txt;
        \item write the line a: 11 12 13 14 15 to c.txt;
        \item concatenate a.txt, b.txt, and c.txt into all.txt.
    \end{itemize}

\item Convert the following binary numbers to decimals:
    \begin{itemize}
        \item \verb|1010010010010000|
        \item \verb|0001000101010001|
        \item \verb|1001010100001100|
        \item \verb|0001010101011011|
    \end{itemize}

\item Convert the binary numbers from the previous exercise to hexadecimals

\item Convert your Student ID number to hexadecimal representation. Hint: convert to binary, and then, to hexadecimal.

% CodeForWin
\item Write a function that returns a missing number in an array of integers ranging from 1 to $n$. For example, given $[3, 2, 1, 5]$ and $n=5$, output \verb|4|.

\item What is the output of the following code? If it breaks at any point, indicate what went wrong.
\begin{verbatim}
    #include <iostream>

    int mystery(int x, int *y) {
        x = x + 10;
        *y = x * 2;
        return x;
    }

    int* mystery2() {
        int x = 50;
        return &x;
    }

    int main() {
        int x = 2, y = 3;
        x = mystery(x, &y);
        std::cout << "(x, y): (" << x << ", " << y << ")" << std::endl;
        int *z = mystery2();
        std::cout << "z: " << *z << std::endl;
    }
\end{verbatim}

\end{enumerate}

\label{r:lastpage}

\end{document}
